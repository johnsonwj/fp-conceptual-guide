
\chapter{Appendix}

\section{Language-Specific Caveats}\label{lang}

Here we will discuss some ``gotchas'' or signficant syntactic deviations from some popular programming languages that might catch the inexperienced reader off-guard if they try to apply the concepts above to that language. Once again, the goal is not to teach a specific language; this is just a collection of warnings and disclaimers that would otherwise have cluttered the main narrative.

\subsection{Scala / \texttt{cats}}\label{lang:scala}

\begin{itemize}
    \item typeclasses via implicits/traits
\end{itemize}

\subsection{Haskell}\label{lang:haskell}

\begin{itemize}
    \item type name translations (Maybe vs Optional, etc)
    \item data vs newtype
    \item newtype for re-implementing typeclasses
    \item dangers with laziness (e.g.\ foldr vs foldl)
    \item reader, writer, state defined in terms of their transformers
\end{itemize}

\subsection{TypeScript / \texttt{fp-ts}}\label{lang:fp-ts}

\begin{itemize}
    \item syntax difficulties
    \item type name translations (Option vs Optional, etc)
\end{itemize}

\subsection{Idris}\label{lang:idris}

\begin{itemize}
    \item labelled interfaces
\end{itemize}

\section{Right Folds From The Left}\label{endo}

Here we are going to look into how to implement a right fold generically, given only a left fold and no other information about the data structure. The idea is that we fold the structure up, from the left, \emph{into a function}, where the resulting function is designed to evaluate the right-most values first. Here's what that looks like:

\begin{lstlisting}[language=pseudoml]
foldr : Foldable t => (a -> b -> b) -> b -> t a -> b
foldr f z t = (foldl foo bar t) z where
    foo = _
    bar = _
\end{lstlisting}

This is pretty much a direct translation of the idea above: we left-fold (somehow) the structure \texttt{t} into a function that evaluates from the right, then kick it off with the given starting value. Now we just need to figure out what to use for \texttt{foo} and \texttt{bar}! Let's start by looking at their types, to see if they give us any clues. We know that \mlil{foldl foo bar t} must be a function, and it should have type \mlil{b -> b}. If we compare that to the type of \mlil{foldl},

\begin{lstlisting}[language=pseudoml]
foldl : Foldable t => (c -> a -> c) -> c -> t a -> c
\end{lstlisting}

\noindent in order to have the correct result type, we must have

\begin{lstlisting}[language=pseudoml]
foo : (b -> b) -> a -> (b -> b)
bar : (b -> b)
\end{lstlisting}

The second one is easy: whenever you need a value that fits that type signature, it almost certainly should be the identity function \mlil{id}. What about \texttt{foo}? Well let's treat it as a function of two arguments:

\begin{lstlisting}[language=pseudoml]
foo : (b -> b) -> a -> (b -> b)
foo g x = _
\end{lstlisting}

At this point, let's consider what gadgets we have available to us. We haven't yet used \mlil{f : a -> b -> b}, and now we also have \mlil{g : b -> b} and \mlil{x : a}. In general, unless you have a good reason, you want to try to use all of the variables you have handy. Interestingly, \mlil{f x} will have type \mlil{b -> b}, so what if we just compose that with \mlil{g}?

\begin{lstlisting}[language=pseudoml]
    foldr : Foldable t => (a -> b -> b) -> b -> t a -> b
    foldr f z t = (foldl foo id t) z where
        foo g x = g . f x
\end{lstlisting}

If you try this out, you'll find that this definition works exactly as we wanted it to! This is actually somewhat amazing, which is a pretty common occurrence with ``type-driven development'' as this method is usually called. We could have arrived at the same result if we sat down and worked out exactly what it means to ``left-fold a structure into a function that executes a right-fold'', but that would have required a lot more noodling.

To be honest, though, we cheated a little bit. Doing the composition in \texttt{foo} the other way around would have typechecked, but produces the wrong results:

\begin{lstlisting}[language=pseudoml]
    notFoldr : Foldable t => (a -> b -> b) -> b -> t a -> b
    notFoldr f z t = (foldl foo id t) z where
        foo g x = f x . g
\end{lstlisting}

If you work out an example, this turns out to look like a right fold\ldots{}but for a reversed input! This is the price of type-driven development: sometimes there is more than one choice to fill in a value for a given type, and the only way to determine which choice is correct is by testing it out yourself. Static type checking is not a substitute for \emph{all} tests!
